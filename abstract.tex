\begin{doublespace}
Hendra virus is a rare disease unique to Australia. The virus originates in protected flying
foxes, but has been directly transferred to horses, and indirectly to humans. Previous
research suggests Hendra virus spillovers are driven by habitat loss and climate, which
force flying foxes out of there typical foraging patterns. Factors that influence flying
fox feeding locations are thus directly related to public health, and species protection.
We examine the impacts of land cover on flying fox feeding locations via tree based models;
using radar data collected near Ballina Byron Gateway Airport in New South Wales Australia
at the end of February 2024. Land cover data is sourced from Google Earth Engine. A
background discussion of several Classifaction and Regression Tree (CART) based algorithms
is included, along with steps to prepare data for modelling.
\end{doublespace}