\begin{doublespace}
Hendra virus is a rare and potentially fatal disease endemic to Australia.
Serological evidence of Hendra virus has been detected in Papua New Guinea as well.
Flying foxes
act as a reservoir host for the Hendra virus, which has been transferred directly to
horses and indirectly to humans. Previous research suggests Hendra virus spillovers are
driven by habitat loss and climate, which can force flying foxes out of there typical
foraging patterns. Factors that influence flying fox feeding locations are thus directly 
related to public health, and species protection. We examine the impacts of land cover on 
flying fox feeding locations via tree based models; using radar data collected near Ballina 
Byron Gateway Airport in New South Wales Australia at the end of February 2024. Land 
cover data is from Google Earth Engine. A background discussion of several
Regression Tree based algorithms is included,
along with steps to prepare data for modelling. The results of our analysis of
foraging sites provide some evidence of associations with man built structures,
flooded vegetation, shrubbery, trees, and water. While land cover alone does not fully
explain the feeding choices of Ballina's flying fox population, there appears to be
some association.
\end{doublespace}